%-----------------------------------------------------------------------------
%
%               Template for sigplanconf LaTeX Class
%
% Name:         sigplanconf-template.tex
%
% Purpose:      A template for sigplanconf.cls, which is a LaTeX 2e class
%               file for SIGPLAN conference proceedings.
%
% Guide:        Refer to "Author's Guide to the ACM SIGPLAN Class,"
%               sigplanconf-guide.pdf
%
% Author:       Paul C. Anagnostopoulos
%               Windfall Software
%               978 371-2316
%               paul@windfall.com
%
% Created:      15 February 2005
%
%-----------------------------------------------------------------------------


\documentclass{sigplanconf}

% The following \documentclass options may be useful:

% preprint      Remove this option only once the paper is in final form.
% 10pt          To set in 10-point type instead of 9-point.
% 11pt          To set in 11-point type instead of 9-point.
% authoryear    To obtain author/year citation style instead of numeric.

\usepackage{amsmath}


\begin{document}

\special{papersize=8.5in,11in}
\setlength{\pdfpageheight}{\paperheight}
\setlength{\pdfpagewidth}{\paperwidth}

\conferenceinfo{PLATEAU '15}{October 26, 2015, Pittsburgh, PA, USA} 
\copyrightyear{2015} 
\copyrightdata{978-1-nnnn-nnnn-n/yy/mm} 
\doi{nnnnnnn.nnnnnnn}

% Uncomment one of the following two, if you are not going for the 
% traditional copyright transfer agreement.

%\exclusivelicense                % ACM gets exclusive license to publish, 
                                  % you retain copyright

%\permissiontopublish             % ACM gets nonexclusive license to publish
                                  % (paid open-access papers, 
                                  % short abstracts)

\titlebanner{banner above paper title}        % These are ignored unless
\preprintfooter{short description of paper}   % 'preprint' option specified.

\title{Modularity's Multitudes}
% \subtitle{Subtitle Text, if any}

\authorinfo{Darya Kurilova \and Chris Martens}
           {Carnegie Mellon University}
           {cmartens@cs.cmu.edu}

\maketitle

\begin{abstract}
This is the text of the abstract.
\end{abstract}

\category{CR-number}{subcategory}{third-level}

% general terms are not compulsory anymore, 
% you may leave them out
\terms
programming languages, usability

\keywords
modularity, programming

\section{Introduction}

XXX purposes of modularity: code reuse, program comprehension, and
interoperability.

\begin{itemize}
  \item {\em Reuse:} program modules should provide general behavior
    that may be used in a variety of contexts.
  \item {\em Comprehension:} partitioning code into modules reveals
    its high-level structure, enabling programmers to understand the system
    as a whole.
  \item {\em Separability:} a module may exist independently from other
    modules, for the sake of separate compilation or distribution.
  \item {\em Interoperability:} module implementations may change with
    time, and code that depends on those modules should not have its
    behavior change when the implementation does.
\end{itemize}

\subsection{Hypothesis}

\begin{quote}
  {\bf Our hypothesis:}
  {\em Modularity}, as a concept in programming, consists of many different
  constructs, each with different usability benefits. In particular,
  \begin{enumerate}
    \item Language designers tend to have very similar goals in mind for
      adding ``modularity'' constructs to their languages, but the concrete
      results that they introduce solve distinct problems.
    \item Modularity constructs intended to enable code reuse,
      comprehension of high-level program structure, and interoperability
      serve orthogonal purposes, i.e. {\em programmers tend to use these
      constructs in ways for which the other constructs would not be
      suitable}.
    \item {\em Novice programmers are often confused about the
      distinctions} between different constructs for modularity and their
      alleged benefits. 
  \end{enumerate}
\end{quote}

\section{A Taxonomy of Modularity Features}

\subsection{Reuse}

\subsection{Comprehension}

\subsection{Separability}

\subsection{Interoperability}

\section{Examples}

Derek Dreyer's thesis:
~\cite{dreyer2005understanding}

MacQueen, ML module system:
~\cite{macqueen1984modules}

Parnas, criteria:
~\cite{parnas1972criteria}

Programming in the large:
~\cite{deremer1976programming}

\section{Conclusion}


\appendix
\section{Appendix Title}

This is the text of the appendix, if you need one.

\acks

Acknowledgments, if needed.

% We recommend abbrvnat bibliography style.

\bibliographystyle{abbrvnat}
\bibliography{main}

% The bibliography should be embedded for final submission.
% XXX uncomment the below/paste from .bbl
% \begin{thebibliography}{}
% \softraggedright

% \bibitem[Smith et~al.(2009)Smith, Jones]{smith02}
% P. Q. Smith, and X. Y. Jones. ...reference text...

% \end{thebibliography}


\end{document}

%                       Revision History
%                       -------- -------
%  Date         Person  Ver.    Change
%  ----         ------  ----    ------

%  2013.06.29   TU      0.1--4  comments on permission/copyright notices

